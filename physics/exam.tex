\documentclass{article}
\usepackage[T2A]{fontenc}

% Hyphenation rules
%--------------------------------------
\usepackage{hyphenat}
\hyphenation{ма-те-ма-ти-ка вос-ста-нав-ли-вать}
%--------------------------------------

% Visuals
\usepackage[russian]{babel}
\usepackage{color}
\usepackage{hyperref}
\hypersetup{
    colorlinks=true,
    linktoc=all,
    linkcolor=blue,
    pdftitle={Выпрями спину},
}
\usepackage[skip=5pt, indent=20pt]{parskip}
\usepackage{wrapfig}
\usepackage{graphicx}

% Math 
\usepackage{amsmath, amsfonts, amsthm, amssymb}
\usepackage{witharrows}
\usepackage{tcolorbox}
\usepackage{siunitx}

% Graphics 
\usepackage{tikz}


\title{Билеты}
\author{Кириенко Михаил 10-7}
\date{2023 - 2024}

\begin{document}
\maketitle
\tableofcontents
\newpage

\section{Билет №1}
Основные положения МКТ, их опытные обоснования. Масса, размеры и скорость молекул. Взаимодействие молекул. 



\begin{tcolorbox}[title=Основные положения МКТ, colback=blue!5!white, colframe=blue!50!black] 
\begin{enumerate}
  \item Все вещества состоят из \textbf{молекул}. 
  \item Все молекулы находятся в состоянии непрырывного, хаотического \textbf{движения} (тепловое движений). 
  \item Все молекулы \textbf{взаимодействуют} друг с другом (приемущественно силы электронного взаимодействия).
\end{enumerate}
\end{tcolorbox}

\subsection{Опытные обоснования}


Опытным обоснованием существания молекул является хотя бы возможность увидеть их с помощью такого прибора, как 
\textit{растровый туннельный микроскоп}. Его принцип действия основан на туннельном эффекте (квантовая механика), 
суть которого заключается в том, что электроны их одного тела способны перескакивать в другое тело даже при отсутствии
электростатического поля. 

\noindent Прочие обоснования существования молекул:
\begin{itemize}
  \item Закон кратный отношений (пример, для образования воды всего используются водород и кислород в отношени 1:8). 
  \item Броуновское движение. 
\end{itemize}

Тепловое движение можно обосновать с помощью такого явления, как \textit{броуновское движение}. Броуновское движение
представляет собой грубое, сильно упрощённое, но глубоко верное отображение теплового движения молекул. Впервые
его наблюдал Роберт Броун, когда стал рассматривать пыльцу, подвешенную в жидкости, под микроскопом и заметил, что частицы 
пыльцы двигались, что можно обосновать только с помощью второго положения МКТ.

\begin{figure}[h]
  \begin{tikzpicture}
    \draw (0, 0) -- (1, 1.5) -- (-1, 2) -- (1, 0.7) -- (2, 2.3) -- (2.7, 1.3) -- (1.4, 0.2) -- (3, 0.2) -- (2.9, 0.7);
    \draw[dashed] (0, 0) -- (2.9, 0.7);

    \filldraw (0, 0) node[anchor=south east]{A} circle (0.05);
    \filldraw (2.9, 0.7) node[anchor=south west]{B} circle (0.05);
    \filldraw (1, 0.7) node[anchor=north west]{1} circle (0.05);
    \filldraw (2, 2.3) node[anchor=south east]{2} circle (0.05);

    \draw (0.9, 0.6) rectangle (2.1, 2.4);
    \draw (4.8, 0) rectangle (7.7, 2.6);

    \draw (2.1, 2.4) -- (4.8, 2.6);
    \draw (2.1, 0.6) -- (4.8, 0);

    \draw (5.3, 0.3) -- (5.4, 0.7) -- (5.9, 0.4) -- (5.8, 1) -- (6.2, 1.7) -- (6.7, 1) -- (7, 1.3) -- (6.9, 2) -- (7.2, 2.3);
    \filldraw (5.3, 0.3) node[anchor=south east]{1} circle (0.05);
    \filldraw (7.2, 2.3) node[anchor=north west]{2} circle (0.05);
  \end{tikzpicture}
\end{figure}

Траектория броуновской частицы является очень сложной и является по сути своей фракталом, а принимает она такую форму
из-за того, что в любой момент времени количествоц ударов молекул по частице с одной стороны будет больше, чем с
другой, что заставит частицу измегить направление движения. 

Еще одним опыт, который подтвердил хаотичность движения частиц, был проведен Жаном Батиcтом Перреном. Он рассмотривал
концентрации броуновских частиц в на определенных высотах в капле воды и заметил, что частицы не скапливались у дна,
а были определенным образом распределенны по капле воды. 

% TODO draw experiment pic and plot (lesson 140 10:00)

\noindent Прочие подтверждения теплового движения молекул:
\begin{itemize}
  \item Диффузия (засолка огурцов). 
  \item Стремление газа занять весь доступный ему объем. 
\end{itemize}

\noindent Обоснования третьего положения МКТ:
\begin{itemize}
  \item Закон Гука (при попытку растянуть пружину возникает сила упргости, которая появляется из-за того, что 
    расстояние между молекулами увеличивается, и она притягиваются друг к другу). 
  \item Деформация тела. 
  \item Сохранение формы твердого тела
\end{itemize}

\subsection{Характеристики молекул}

В своем опыте Жан Батист Перрен (упомянутый ранее) выяснил, что масса молекулы кислорода $m_0 = 5 * 10^{-26} \si{kg}$,
все остальные молекулы в свою очередь имею массу сравнимую с этом значением. 

Однако измерение массы молекулы в
килограммах не является удобным из-за их маленького размера, поэтому было принято решение считать массы молекул
относительно какого-то определенного вещества, которым являлся поначалу водород, но из-за его неспособности реагировать
с определенными элементами эталон позже был изменен на $\frac{1}{12}$ массы молекулы углерода-12 (так сохранялась 
обратная совместивость). Такие образом $1 \text{а.е.м.} = \frac{1}{12}m_0(\text{C}) = 1.66 * 10^{-27} \si{kg}$, 
где а.е.м. - \textit{атомная единица массы}.

Также используется такое понятие, как \textit{относительная молекулярная масса (\textmu)}, которым называют физическую величину
равную отношению массы молекулы этого вещества к массе $\frac{1}{12}$ молекулы углерода-12.
\[
  \mu = \frac{m_0}{\frac{1}{12}m_0(\text{C})}
\]

\textit{Количество вещества} ($\nu$) называют отношение числа молекул $N$ в данном теле к числу атомов $N_A$ в 12г углерода. 
Измеряется количество вещества в молях, \textit{моль} - это количество вещества, содержащего столько же молекул, 
сколько атомов содержится в 12г углерода. 

\begin{align*}
  \colorbox{yellow}{$\nu = \frac{N}{N_{A}} = \frac{m}{\mu}$}
\end{align*}

$N_A$ в свою очередь является \textit{постоянной Авогадро} и обозначает количество молекул в одном моле вещетсва, приблезительно
равно $6.022 * 10^{23} \si{mol^{-1}}$. 

\textit{Молярной массой} ($\text{M}$, $\mu$) называют массу вещества взятого в количество одного моля. 
\[
  \mu = {m_0} * {N_A}
\]

Размер молекулы можно увидеть в микроскоп, например, размер молекулы водорода составляет $1,4 * 10^{-8} \si{cm}$, 
также размеры молекулы можно вычислить:
% TODO add grid
\begin{gather*}
  V_0 = \frac{V}{N} \\
  N = \nu * N_A \\
  \nu = \frac{m}{\mu} \\
  V_0 = \frac{V}{\frac{m}{\mu}N_A} = \frac{V\mu}{mN_A}\\ 
  m = \rho V \implies \frac{V}{m} = \frac{1}{\rho} \\
  V_0 = \frac{\mu}{\rho N_A} \\
  d^3 = V_0 \\
  d = \sqrt[3]{\frac{\mu}{\rho N_A}}
\end{gather*}



\subsection{Межмолекулярное взаимодействие}

График для силы в зависимости от расстояния между молекулярами.
1/r12 - 1/r6 

\subsection{Глоссарий}

Молекулярео-кинетической теорией(МКТ) называют учение о строении и свойствах вещества на основе представления о 
существовании атомов и молекул как наименьших частиц химических веществ. 

Молекулой называется мельчайшая частица вещества, сохраняющая его химические, но не физические свойства.

Броуновская частица

Тепловое движение

Диффузия

Молярная масса

Постоянная авагадро

Количество вещества

Молярная масса

Относительная молекулярная масса

Размео молекулы

m 

m0 

M 

Mr 

N 

NA 

nu 

n = N / V 

\subsection{Фанфакты}
\begin{itemize}
  \item Ангстрем 
\end{itemize}

\end{document}
